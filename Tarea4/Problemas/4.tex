\section{Problema 4.} Resuelva:
$$\frac{\partial^{2} y}{\partial t^{2}}=c^{2} \frac{\partial^{2} y}{\partial x^{2}}+\frac{F_{o}}{\rho}$$

Con las condiciones: $0<x<L, t>0$, sujeta a 
$$y(0, t)=y(L, t)=y(x, 0)=\frac{\partial y}{\partial t}(x, 0)=0$$

\begin{solution}
	Aplicando la transformada de Laplace (en términos de $f(t)$), tenemos: 
	\begin{align*}
		\Laplace{\frac{\partial^{2} y}{\partial t^{2}}}&=c^{2}\Laplace{ \frac{\partial^{2} y}{\partial x^{2}}}+\frac{F_{o}}{\rho}\Laplace{1}\\
		s^2\widecheck{y}(x,s)-sy(x,0)-y'(x,0)&= c^2\widecheck{y}''(x,s)+\frac{F_o}{\rho}\cdot \frac{1}{s}\\
		s^2\widecheck{y}(x,s)&= c^2\widecheck{y}''(x,s)+\frac{F_o}{\rho}\cdot\frac{1}{s}
		\intertext{Implica:}
		c^2\widecheck{y}''(x,s)-s^2\widecheck{y}(x,s) &= -\frac{F_o}{\rho }\cdot \frac{1}{s}\\
		\widecheck{y}''(x,s)-\frac{s^2}{c^2}\widecheck{y}(x,s) &= -\frac{F_o}{c^2\rho s}
		\intertext{Substituimos $k_1=s^2/c_2$ y $k_2=-F_o/c^2\rho s$:}
		\widecheck{y}''(x,s)-k_1\widecheck{y}(x,s) &= k_2
		\intertext{Usamos una notación más cómoda:}
		\widecheck{y}''-k_1\widecheck{y} &= k_2
	\end{align*}

\linea 

La forma de la solución es la siguiente: 
$$\widecheck{y}(x,s)=\widecheck{y}_H(x,s)+\widecheck{y}_P(x,s)$$

\linea 

Caso homógeneo

$$\widecheck{y}_H''-k_1\widecheck{y}_H = 0$$

En donde su solución es: 

$$\widecheck{y}_H(x,s)=Ae^{\sqrt{k_1}x}+Be^{-\sqrt{k_1}x}$$

\linea 

Caso particular 

$$\widecheck{y}_P''-k_1\widecheck{y}_P = k_2$$

Por el método de coeficientes intedeterminados, se propone $\widecheck{y}_P(x,s)=c$. Es decir:  

\begin{align*}
	\widecheck{y}_P''-k_1\widecheck{y}_P = \frac{\partial^2 x}{\partial x^2} c -k_1(c)  = 0 -k_1c.
\end{align*}
Por lo que, implica: 
$$-k_1c=k_2\implies c=-\frac{k_2}{k_1}.$$
Es decir, la solución del caso particular: 

$$\widecheck{y}_P(x,s)=-\frac{k_2}{k_1}.$$

\linea 

La solución del caso general: 
\begin{align*}
	\widecheck{y}(x,s) &=\widecheck{y}_H(x,s)+\widecheck{y}_P(x,s)\\
										  &= Ae^{\sqrt{k_1}x}+Be^{-\sqrt{k_1}x}-\frac{k_2}{k_1}.
\end{align*}
Aplicando las condiciones, $\widecheck{y}(0,s)=$, entonces:
$$\widecheck{y}(0,s) = A+B -\frac{k_2}{k_1}=0\implies B=\frac{k_2}{k_1}-A.$$

Lo que quiere decir: 
\begin{align*}
	\widecheck{y}(x,s)&=Ae^{\sqrt{k_1}x}+\left(\frac{k_2}{k_1}-A\right)e^{-\sqrt{k_1}x}-\frac{k_2}{k_1}\\
											&= A\left(e^{\sqrt{k_1}x}-e^{-\sqrt{k_1}x}\right)+\frac{k_2}{k_1}\left(e^{-\sqrt{k_1}x}-1\right)\\
											&= 2A\senh(\sqrt{k_1}x)+\frac{k_2}{k_1}\left(e^{-\sqrt{k_1}x}-1\right)
	\intertext{Aplicando $\widecheck{y}(L,s)=0$, tenemos:}
	\widecheck{y}(L,s)&= 2A\senh(\sqrt{k_1}L)+\frac{k_2}{k_1}\left(e^{-\sqrt{k_1}L}-1\right)=0
\end{align*}
Implica:
$$A=\frac{k_2\left(1-e^{-\sqrt{k_1}L}\right)}{2k_1\senh(\sqrt{k_1}L)}$$

Entonces:

\begin{align*}
	\widecheck{y}(x,s)&= \left(\frac{k_2\left(1-e^{-\sqrt{k_1}L}\right)}{k_1\senh(\sqrt{k_1}L)}\right)\senh(\sqrt{k_1}x)+\frac{k_2}{k_1}\left(e^{-\sqrt{k_1}x}-1\right)
	\intertext{Sustituyendo a las constantes originales:}
	&= \left(\frac{F_o}{\rho s^3}\right)\left(\frac{1-e^{-\frac{s}{c}L}}{\senh(\frac{s}{c}L)}\right)\senh\left(\frac{s}{c}x\right)+\left(\frac{F_o}{\rho s^3}\right)\left(e^{-\frac{s}{c}x}-1\right)\\
	&=  \left(\frac{F_o}{\rho s^3}\right)\left(\frac{1-e^{-\frac{s}{c}L}}{\senh(\frac{s}{c}L)}\right)\senh\left(\frac{s}{c}x\right)-\left(\frac{F_o}{\rho s^3}\right)\left(1-e^{-\frac{s}{c}x}\right)
\end{align*}

Entonces, 

\begin{align*}
	y(x,s)&=\left(\frac{F_o}{\rho s^3}\right)\iLaplace{\left(\frac{1-e^{-\frac{s}{c}L}}{\senh(\frac{s}{c}L)}\right)\senh\left(\frac{s}{c}x\right)-\left(1-e^{-\frac{s}{c}x}\right)}\\
	&= \left(\frac{F_o}{\rho s^3}\right)\iLaplace{\frac{(1-e^{-\frac{s}{c}L})\senh(\frac{s}{c}x)-\senh(\frac{s}{c}L)+\senh(\frac{s}{c}L)e^{-\frac{s}{c}x}}{\senh(\frac{s}{c}L)}}\\
	&=\left(\frac{F_o}{\rho s^3}\right)\iLaplace{\frac{(1-e^{-\frac{s}{c}L})\senh(\frac{s}{c}x)-\senh(\frac{s}{c}L)+\senh(\frac{s}{c}L)e^{-\frac{s}{c}x}}{\senh(\frac{s}{c}L)}}
	\intertext{Luego del desarrollo algebraico de la forma exponencial de los senos hipérbolicos:}
	&=-\left(\frac{F_o}{\rho s^3}\right)\iLaplace{\frac{\senh(s\frac{x-L}{c})}{\senh(s\frac{L}{c})}} \text{ (resolver con convolución.)}
\end{align*}


\begin{tcolorbox}[colback=gray!15,colframe=black!1!black,title=Caso no acotado (me había olvidado que era acotado y ya lo había hecho :'v)]
Ahora notamos que, mientras $e^{\sqrt{k_1}x}\to \infty$ y $x\to \infty$, entonces $A=0$. Por lo tanto: 

\begin{align*}
	\widecheck{y}(x,s)&=\left(\frac{k_2}{k_1}\right)e^{-\sqrt{k_1}x}-\frac{k_2}{k_1}
	\intertext{Sustituimos a sus variables originales $k_1$ y $k_2$: }
										&= \left(\frac{-F_o/(c^2\rho s)}{s^2/c^2}\right)e^{-\sqrt{s^2/c^2}x}-\frac{-F_o/(c^2\rho s)}{s^2/c^2}\\
										&= -\left(\frac{F_o}{\rho s^3}\right)e^{-\frac{s}{c}x}+\left(\frac{F_o}{\rho s^3}\right)
\end{align*}

Aplicando la trasnformada inversa de Laplace, se tiene: 

$$y(x,t)=\iLaplace{\widecheck{y}(x,s)}=\iLaplace{\left(\frac{F_o}{\rho s^3}\right)-\left(\frac{F_o}{\rho s^3}\right)e^{-\frac{s}{c}x}}=$$
$$=\frac{F_o}{\rho}\iLaplace{\frac{1}{s^3}}-\frac{F_o}{\rho}\iLaplace{\frac{1}{s^3}\cdot e^{-\frac{s}{c}x}}=$$
$$=\frac{F_o}{\rho}\cdot \frac{t^2}{2}-\frac{F_o}{2\rho}\left(t-\frac{x}{c}\right)^2H\left(t-\frac{x}{c}\right)$$
\end{tcolorbox}


\end{solution}