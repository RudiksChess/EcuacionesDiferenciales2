% !TeX spellcheck = <none>
\section{Problema 3.} 
Resuelva: $$\frac{\partial u}{\partial t}=4 \frac{\partial^{2} u}{\partial x^{2}},\quad t>0$$
Con las condiciones:
 $$u(0, t)=0, \quad u(3, t)=0,\quad u(x, 0)=10 \sen 2 \pi x-6 \sen 4 \pi x$$
\begin{solution}
	Aplicando la transformada de Laplace (en términos de $f(t)$), tenemos: 
	\begin{align*}
	\Laplace{\frac{\partial u}{\partial t}}&=4 \Laplace{\frac{\partial^{2} u}{\partial x^{2}}}\\
	s\widecheck{u}(x,s)-u(x,0) &= 4\widecheck{u}''(x,s)
	\intertext{Aplicando una de las condiciones iniciales:}
	s\widecheck{u}(x,s)-\left(10 \sen 2 \pi x-6 \sen 4 \pi x\right) &= 4\widecheck{u}''(x,s)\\
	4\widecheck{u}''(x,s) -s\widecheck{u}(x,s)&= 6 \sen 4 \pi x -10 \sen 2 \pi x\\
	\widecheck{u}''(x,s) -\frac{s}{4}\widecheck{u}(x,s)&= \frac{3}{2} \sen 4 \pi x    -\frac{5}{2} \sen 2 \pi x
	\intertext{Usando una notación más cómoda:}
		\widecheck{u}''-\frac{s}{4}\widecheck{u}&= \frac{3}{2} \sen 4 \pi x -\frac{5}{2} \sen 2 \pi x
	\end{align*}

\linea 

Se propone tratar el problema por el método del caso homógeneo y caso particular: 
$$\widecheck{u}(x,s)=\widecheck{u}_H(x,s)+\widecheck{u}_P(x,s)$$

\linea 

Caso homógeneo :
$$\widecheck{u}''-\frac{s}{4}\widecheck{u}= 0$$
Haciendo una sustitución: 

$$m^2-\frac{s}{4}=0\implies m=\pm\sqrt{\frac{s}{4}}$$

La solución del caso homógeneo: 

$$\widecheck{u}(x,s)= Ae^{+\sqrt{s/4}x}+Be^{-\sqrt{s/4}x}$$

\linea 

Caso particular: 

$$	\widecheck{u}''-\frac{s}{4}\widecheck{u}= \frac{3}{2} \sen 4 \pi x -\frac{5}{2} \sen 2 \pi x$$

Se propone tratar el problema por el método de coeficientes indeterminados, proponiendo: 
$$\widecheck{u}_P(x,s)=A\sen4\pi x+B\cos4\pi x+C\sen2\pi x+D\cos2\pi x$$
En donde, 
$$\widecheck{u}'_P(x,s)=4\pi A\cos4\pi x-4\pi B\sen4\pi x+2\pi C \cos2\pi x-2\pi D\sen2\pi x$$
$$\widecheck{u}''_P(x,s)=-16\pi^2 A \sen4\pi x-16\pi^2 B \cos4\pi x-4\pi^2 C \sen2\pi x-4\pi^2 D\cos2\pi x$$

Haciendo una sustitución en la expresión propuesta: 

\begin{align*}
		\widecheck{u}''-\frac{s}{4}\widecheck{u} &= (-16\pi^2 A \sen4\pi x-16\pi^2 B \cos4\pi x-4\pi^2 C \sen2\pi x-4\pi^2 D\cos2\pi x)-\\
		&-\frac{s}{4}(A\sen4\pi x+B\cos4\pi x+C\sen2\pi x+D\cos2\pi x)\\
		&= -\left(16\pi +\frac{s}{4}\right)A\sin 4\pi x-\left(16\pi +\frac{s}{4}\right) B \cos4\pi x-\\
		&-\left(4\pi +\frac{s}{4}\right)C \sen2\pi x-\left(4\pi +\frac{s}{4}\right)D \cos2\pi x
\end{align*}
Entonces, 
\begin{align*}
	\begin{split}
		-\left(16\pi^2 +\frac{s}{4}\right)A\sen 4\pi x-\left(16\pi^2+\frac{s}{4}\right) B \cos4\pi x-\\-\left(4\pi^2+\frac{s}{4}\right)C \sen2\pi x-\left(4\pi^2 +\frac{s}{4}\right)D \cos2\pi x 
	\end{split} &= \frac{3}{2} \sen 4 \pi x -\frac{5}{2} \sen 2 \pi x
\intertext{Que es lo mismo que:}
	\begin{split}
		-\left(64\pi^2 +s\right)A\sen 4\pi x-\left(64\pi^2 +s\right) B \cos4\pi x-\\-\left(16\pi^2 +s\right)C \sen2\pi x-\left(16\pi^2 +s\right)D \cos2\pi x 
	\end{split} &= \frac{3}{2} \sen 4 \pi x -\frac{5}{2} \sen 2 \pi x
\end{align*}
Nos damos cuenta que podemos asumir: $B=0=D$, entonces:

$$-\left(64\pi^2 +s\right)A\sen 4\pi x-\left(16\pi^2+s\right)C \sen2\pi x=\frac{3}{2} \sen 4 \pi x -\frac{5}{2} \sen 2 \pi$$

Es decir,

$$-\left(64\pi^2+s\right)A=\frac{3}{2}\qquad \text{ y } \qquad-\left(16\pi^2 +s\right)C=-\frac{5}{2}  $$
Entonces: 
$$A=-\frac{3}{2(64\pi^2 +s)}=-\frac{6}{64\pi^2 +s} \qquad \text{ y } \qquad C=\frac{5}{2(16\pi^2 +s)}=\frac{10}{16\pi^2 +s}$$

Por lo tanto, la solución del caso particular es: 

$$\widecheck{u}_P(x,s)=-\left(\frac{6}{64\pi^2 +s}\right)\sen4\pi x+\left(\frac{10}{16\pi^2 +s}\right)\sen2\pi x$$

\linea 

La solución general: 
\begin{align*}
	\widecheck{u}(x,s)&=\widecheck{u}_H(x,s)+\widecheck{u}_P(x,s)\\
									      &=  Ae^{+\sqrt{s/4}x}+Be^{-\sqrt{s/4}x} +\left(\frac{10}{16\pi^2 +s}\right)\sen2\pi x-\left(\frac{6}{64\pi^2 +s}\right)\sen4\pi x
\end{align*}
Considerando las condiciones, $\widecheck{u}(0,s)=0$ y $\widecheck{u}(3,s)=0$, tenemos: 
$$\widecheck{u}(0,s)= A+B =0\implies B=-A$$

Es decir: 

$$\widecheck{u}(x,s)=C\cosh{\sqrt{\frac{s}{4}}x}+\left(\frac{10}{16\pi^2 +s}\right)\sen2\pi x-\left(\frac{6}{64\pi^2 +s}\right)\sen4\pi x$$

Aplicando la última condición: 
$$\widecheck{u}(3,s)=C\cosh{\sqrt{\frac{s}{4}}\cdot3}=0\implies C=0$$

Por lo tanto, la solución general final es: 

$$\widecheck{u}(x,s)=\left(\frac{10}{16\pi^2 +s}\right)\sen2\pi x-\left(\frac{6}{64\pi^2 +s}\right)\sen4\pi x$$

Aplicando la transfromada inversa de Laplace:

$$u(x,t)=\iLaplace{u(x,s)}(t)=\iLaplace{\left(\frac{10}{16\pi^2 +s}\right)\sen2\pi x-\left(\frac{6}{64\pi^2 +s}\right)\sen4\pi x}=$$
$$=10\iLaplace{\left(\frac{1}{s-(-16\pi^2) }\right)}\sen2\pi x-6\iLaplace{\left(\frac{1}{s-(-64\pi^2) }\right)}\sen4\pi x=$$
$$=10e^{-16\pi^2t}\sen2\pi x-6e^{-64\pi^2t}\sen4\pi x $$




\end{solution}