\section{Problema 1}
\subsection{} Escriba la solución del problema de Dirichlet para el primer cuadrante si
$$
\begin{array}{c}
u(x, 0)=0, x>0 \\
u(0, y)=g(y), y>0
\end{array}
$$
\begin{solution}
El problema de Dirichlet en el primer cuadrante hace referencia a: 
$$\nabla^2(x,y)=\frac{\partial ^2 u}{\partial x^2}+\frac{\partial ^2 u}{\partial y^2}=0 \qquad x>0,y>0 $$

\linea 

Procedemos por separación de variables: 
$$u(x,y)=X(x)\cdot Y(y)= X\cdot Y$$

\begin{gather*}
    \implies X''Y+Y''X = 0 \implies X''Y=-Y''X \implies \frac{X''}{X} = -\frac{Y''}{Y}=\omega^2\\
\end{gather*}
Por lo tanto, 
\begin{gather}
     X''-\omega^2 X=0 
\end{gather}

\begin{gather}
     Y''+\omega^2 Y=0
\end{gather}

\linea 

Para (1) la solución de la EDO es, 

$$X_\omega(x)= A_\omega e^{-\omega x}+ B_\omega e^{\omega x}$$
Nótese para que se generen soluciones cuando $x\to \infty$ (que se mantenga acotado), es necesario eliminar el término $B$. Por lo tanto, 
$$X_\omega(x)= A_\omega e^{-\omega x}$$

\linea 

Para (2) la solución de la EDO es, 

$$Y_\omega (y)= C_\omega \cos(\omega y)+D_\omega\sin(\omega y), \qquad u(x,0)=0$$
$$\implies Y_\omega(0)= C_\omega\cos(0)+D_\omega\sin(0)=0\implies C_\omega=0$$
$$\implies Y_\omega(y)=D_\omega\sin(\omega y)$$

\linea 

Es decir que tenemos,
$$X_\omega(x)\cdot Y_\omega(y)=A_\omega e^{-\omega x}\cdot D_\omega\sin(\omega y)=F_\omega e^{-\omega x}\sin(\omega y), \quad \omega >0$$

\linea 

Por el principio de superposición, 
$$u(x,y)=\int_0^\infty F_\omega e^{-\omega x}\sin(\omega y) d \omega $$

Aplicando la condición $u(0,y)=g(y)$, 

$$u(0,y)=g(y)=\int_0^\infty F_\omega\sin(\omega y) d \omega $$
En donde $F_\omega$ es, 
$$F_\omega= \frac{2}{\pi}\int_0^\infty g(\psi)\sin(\psi y)d\psi $$

\linea 

La solución, 
$$u(x,y)=\int_0^\infty \left[\frac{2}{\pi}\int_0^\infty g(\psi)\sin(\psi y)d\psi\right] e^{-\omega x}\sin(\omega y) d \omega $$

\end{solution}










%--------------------
\subsection{} Escriba la solución del problema de Dirichlet para el primer cuadrante si
$$
\begin{array}{l}
u(x, 0)=f(x), x>0 \\
u(0, y)=g(y), y>0
\end{array}
$$

\begin{solution}
El problema de Dirichlet en el primer cuadrante hace referencia a: 
$$\nabla^2(x,y)=\frac{\partial ^2 u}{\partial x^2}+\frac{\partial ^2 u}{\partial y^2}=0 \qquad x>0,y>0 $$
 
Nos damos cuenta que tenemos dos casos, que en el resultado final se sumarán: 

\begin{enumerate}
    \item $$\begin{array}{l}
u(x, 0)=0, x>0 \\
u(0, y)=g(y), y>0
\end{array}
$$ Nos damos cuenta que este caso es el del inciso \textbf{1.1}. Así que no lo resolveremos. 
\item $$\begin{array}{l}
u(x, 0)=f(x), x>0 \\
u(0, y)=0, y>0
\end{array}
$$

\end{enumerate}

\linea 

Procedemos por separación de variables: 
$$u(x,y)=X(x)\cdot Y(y)= X\cdot Y$$

\begin{gather*}
    \implies X''Y+Y''X = 0 \implies X''Y=-Y''X \implies -\frac{X''}{X} = \frac{Y''}{Y}=\omega^2\\
\end{gather*}
\setcounter{equation}{0}
Por lo tanto, 

\begin{gather}
     Y''-\omega^2 Y=0
\end{gather}
\begin{gather}
     X''+\omega^2 X=0 
\end{gather}


\linea 

Para (1) la solución de la EDO es, 

$$Y_\omega(y)= A_\omega e^{-\omega y}+ B_\omega e^{\omega y}$$
Nótese para que se generen soluciones cuando $y\to \infty$ (que se mantenga acotado), es necesario eliminar el término $B$. Por lo tanto, 
$$Y_\omega(y)= A_\omega e^{-\omega y}$$

\linea 

Para (2) la solución de la EDO es, 

$$X_\omega (x)= C_\omega \cos(\omega x)+D_\omega\sin(\omega x), \qquad u(0,y)=0$$
$$\implies X_\omega(0)= C_\omega\cos(0)+D_\omega\sin(0)=0\implies C_\omega=0$$
$$\implies X_\omega(x)=D_\omega\sin(\omega x)$$

\linea 

Es decir que tenemos,
$$X_\omega(x)\cdot Y_\omega(y)=A_\omega e^{-\omega y}\cdot D_\omega\sin(\omega x)=F_\omega e^{-\omega y}\sin(\omega x), \quad \omega >0$$

\linea 

Por el principio de superposición, 
$$u(x,y)=\int_0^\infty F_\omega e^{-\omega y}\sin(\omega x) d \omega $$

Aplicando la condición $u(x,0)=f(x)$, 

$$u(x,0)=f(x)=\int_0^\infty F_\omega\sin(\omega x) d \omega $$
En donde $F_\omega$ es, 
$$F_\omega= \frac{2}{\pi}\int_0^\infty f(\psi)\sin(\psi x)d\psi $$

\linea 

La solución del \textbf{segundo caso}, 
$$u(x,y)=\int_0^\infty \left[\frac{2}{\pi}\int_0^\infty f(\psi)\sin(\psi x)d\psi\right] e^{-\omega y}\sin(\omega x) d \omega $$

\linea

\linea 

La solución final, sumando los 2 casos:

$$u(x,y)= \int_0^\infty \left[\frac{2}{\pi}\int_0^\infty g(\psi)\sin(\psi y)d\psi\right] e^{-\omega x}\sin(\omega y) d \omega+$$
$$+\int_0^\infty \left[\frac{2}{\pi}\int_0^\infty f(\psi)\sin(\psi x)d\psi\right] e^{-\omega y}\sin(\omega x) d \omega$$

\end{solution}