\documentclass[a4paper,12pt]{article}
\usepackage[top = 2.5cm, bottom = 2.5cm, left = 2.5cm, right = 2.5cm]{geometry}
\usepackage[T1]{fontenc}
\usepackage[utf8]{inputenc}
\usepackage{multirow} 
\usepackage{booktabs} 
\usepackage{graphicx}
\usepackage[spanish]{babel}
\usepackage{setspace}
\setlength{\parindent}{0in}
\usepackage{float}
\usepackage{fancyhdr}
\usepackage{amsmath}
\usepackage{amssymb}
\usepackage{amsthm}
\usepackage{natbib}
\usepackage{graphicx}
\usepackage{subcaption}
\usepackage{booktabs}
\usepackage{etoolbox}
\usepackage{apalike}
\usepackage{minibox}
\usepackage{hyperref}
\usepackage{xcolor}
\usepackage{tcolorbox}
\usepackage{tikz}
\usepackage{pdfpages}
\usepackage{mathabx}
\usetikzlibrary{patterns}
\newcommand{\linea}{\noindent\rule{\textwidth}{1pt}}
\AtBeginEnvironment{align}{\setcounter{equation}{0}}
\newenvironment{solution}
  {\renewcommand\qedsymbol{$\square$}\begin{proof}[\textcolor{blue}{Solución}]}
  {\end{proof}}
\newcommand{\Laplace}[1]{\mathcal{L}\left\{#1\right\}}
\newcommand{\iLaplace}[1]{\mathcal{L}^{-1}\left[#1\right]}



\pagestyle{fancy}

\fancyhf{}

\lhead{\footnotesize Ecuaciones Diferenciales 2}
\rhead{\footnotesize  Rudik Roberto Rompich}
\cfoot{\footnotesize \thepage}

\begin{document}
    \thispagestyle{empty} 
    \begin{tabular}{p{15.5cm}}
    \begin{tabbing}
    \textbf{Universidad del Valle de Guatemala} \\
    Departamento de Matemática\\
    Licenciatura en Matemática Aplicada\\\\
   \textbf{Estudiante:} Rudik Roberto Rompich\\
   \textbf{E-mail:} \textcolor{blue}{ \href{mailto:rom19857@uvg.edu.gt}{rom19857@uvg.edu.gt}}\\
   \textbf{Carné:} 19857
    \end{tabbing}
    \begin{center}
        MM2030 - Ecuaciones Diferenciales 2 - Catedrático: Dorval Carías\\
        \today
    \end{center}\\
    \hline
    \\
    \end{tabular} 
    \vspace*{0.3cm} 
    \begin{center} 
    {\Large \bf Tarea 6
} 
        \vspace{2mm}
    \end{center}
    \vspace{0.4cm}
    %---------------------------
%\begin{tcolorbox}[colback=gray!15,colframe=black!1!black,title=A nice heading]
%\end{tcolorbox}

%\fbox{lol}
%---------------------------

Resuelva los  problemas  con  valores en  la  frontera  que  se  presentan  a continuación, utilizando el método a su elección.
\section{Problema 1.} Resuelva: 
$$\frac{\partial u}{\partial t^2}=a^2\frac{\partial^2}{\partial x^2}, \qquad x>0, t,0$$
Sujeta a: $u(0,t)=f(t), \quad u(x,0)=0, \lim_{x\to \infty}u(x,t)=0, \frac{\partial u}{\partial t}(x,0)=0$.
\begin{solution}
Aplicando la transformada de Laplace: 
\begin{align*}
	\Laplace{\frac{\partial u}{\partial t^2}}&=a\Laplace{\frac{\partial^2}{\partial x^2}}\\
	s^2 \widehat{u}(x,s)-su(x,0)-u'(x,0) &= a^2 \widehat{u}''(x,s)\\
	s^2  \widehat{u} -0 -0 &= a^2 \widehat{u}''(x,s)
	\intertext{Aplicando las condiciones: }
	s^2  \widehat{u} &= a^2 \widehat{u}''\\
	a^2 \widehat{u}'' -s^2  \widehat{u} &= 0\\
	 \widehat{u}'' -\left(\frac{s}{a}\right)^2  \widehat{u} &= 0
\end{align*}

\linea 

Resolviendo la EDO: 

$$\widehat{u}(x,s)=Ae^{\frac{s}{a}x}+Be^{-\frac{s}{a}x}$$
Aplicando $\lim_{x\to\infty}\widehat{u}(x,s)=0$, $B$ se hace 0:
$$\lim_{x\to\infty}\widehat{u}(x,s)=\underbrace{\lim_{x\to\infty} \left(Ae^{\frac{s}{a}x}+Be^{-\frac{s}{a} x}\right)}_{\text{A=0, para que se cumpla la condición.}}=0$$
Entonces 
$$\widehat{u}(x,s)=Be^{\frac{s}{a}x}$$
Aplicando $\widehat{u}(0,s)=\widehat{f}(s)$, tenemos: 
$$\widehat{u}(0,s)=B=\widehat{f}(s)$$
Por lo que tenemos: 
$$\widehat{u}(x,s)=\widehat{f}(s)e^{-\frac{s}{a}x}$$
$$u(x,s)=\iLaplace{\widehat{u}(x,s)}=\iLaplace{\widehat{f}(s)e^{-\frac{s}{a}x}}=f\left(t-\frac{x}{a}\right)H\left(t-\frac{x}{a}\right).$$
\end{solution}

\section{Problema 2.} Encuentre la solución acotada de $$\frac{\partial u}{\partial t}=\frac{\partial^{2} u}{\partial x^{2}}$$

Con las condiciones: $ x>0,  t>0$, y tal que $$u(0, t)=1, \qquad u(x, 0)=0$$

\begin{tcolorbox}[colback=gray!15,colframe=black!1!black,title=Nota 1]
	Debido a que no hay ninguna cota en $x$, se impondrá una nueva condición (basado en los principios físicos que gobiernan la ecuación de calor): 
	$$\lim_{x\to\infty}u(x,t)=0, \quad t>0.$$
\end{tcolorbox}

\begin{solution}
	Aplicando la transformada de Laplace (en términos de $f(t)$), tenemos: 
	\begin{align*}
		\frac{\partial u}{\partial t}&=\frac{\partial^{2} u}{\partial x^{2}}\\
		\Laplace{\frac{\partial u}{\partial t}}&=\Laplace{\frac{\partial^{2} u}{\partial x^{2}}}\\
		s\widecheck{u}(x,s)-u(x,0) &= \widecheck{u}''(x,s)
		\intertext{Aplicando $u(x, 0)=0\implies \widecheck{u}(x, 0)=0$:}
		s\widecheck{u}(x,s)&= \widecheck{u}''(x,s)
		\intertext{Implica:}
		\widecheck{u}''(x,s)-s\widecheck{u}(x,s) &= 0
		\intertext{Con una notación más cómoda:}
		\widecheck{u}''-s\widecheck{u} &= 0
		\intertext{Resolviendo la EDO:}
		\widecheck{u}(x,s) &= Ae^{(-\sqrt{s}x)}+Be^{(\sqrt{s}x)}	
		\intertext{Por la \textbf{Nota 1}, sabemos $\lim_{x\to\infty}u(x,t)=0\implies \lim_{x\to\infty}\widecheck{u}(x,s)=0$. Por lo tanto, $B$ debe ser 0:}
		\widecheck{u}(x,s) &= Ae^{(-\sqrt{s}x)}
		\intertext{Aplicando la condición $u(0, t)=1\implies \widecheck{u}(0,s)=1/s$:}	
		\widecheck{u}(0,s) &= A=\frac{1}{s}
		\intertext{Entonces:}
		\widecheck{u}(x,s) &= \frac{1}{s}e^{(-\sqrt{s}x)}
		\intertext{La solución es:}
		u(x,t)&=  \iLaplace{u(x,s)}(t)\\ 
		&=\iLaplace{\frac{1}{s}e^{(-\sqrt{s}x)}}\\
		&= erfc\left(\frac{x}{2\sqrt{t}}\right)
\end{align*}
\end{solution}
\section{Problema 3.} 
Resuelva: $$\frac{\partial u}{\partial t}=4 \frac{\partial^{2} u}{\partial x^{2}},\quad t>0$$
Con las condiciones:
 $$u(0, t)=0, \quad u(3, t)=0,\quad u(x, 0)=10 \sen 2 \pi x-6 \sen 4 \pi x$$
\begin{solution}
	Aplicando la transformada de Laplace (en términos de $f(t)$), tenemos: 
	\begin{align*}
	\Laplace{\frac{\partial u}{\partial t}}&=4 \Laplace{\frac{\partial^{2} u}{\partial x^{2}}}\\
	s\widecheck{u}(x,s)-u(x,0) &= 4\widecheck{u}''(x,s)
	\intertext{Aplicando una de las condiciones iniciales:}
	s\widecheck{u}(x,s)-\left(10 \sen 2 \pi x-6 \sen 4 \pi x\right) &= 4\widecheck{u}''(x,s)\\
	4\widecheck{u}''(x,s) -s\widecheck{u}(x,s)&= 6 \sen 4 \pi x -10 \sen 2 \pi x\\
	\widecheck{u}''(x,s) -\frac{s}{4}\widecheck{u}(x,s)&= \frac{3}{2} \sen 4 \pi x -\frac{5}{2} \sen 2 \pi x
	\intertext{Usando una notación más cómoda:}
		\widecheck{u}''-\frac{s}{4}\widecheck{u}&= \frac{3}{2} \sen 4 \pi -\frac{5}{2} \sen 2 \pi x
	\end{align*}
\end{solution}
\section{Problema 4.} Resuelva:
$$\frac{\partial u}{\partial t} = \frac{\partial ^2 u}{\partial x^2}, \qquad -\infty<x<\infty, t>0.$$
Sujeta a: $u(x,0)=e^{-|x|}$

\begin{solution}
	Aplicando transformada de Fourier. 
	\begin{align*}
		\mathcal{F}\left\{\frac{\partial u}{\partial t}\right\}=\mathcal{F}\left\{\frac{\partial ^2 u}{\partial x^2}\right\}\\
		\hat{u}(w,t)&=(iw)^2\hat{u}(w,t) \\
	\hat{u}(w,t)	&= -w^2\hat{u}(w,t)
	\end{align*}
Entonces, tenemos: 
$$\hat{u}'(w,t)+w^2\hat{u}(w,t)=0.$$
La solución de la EDO: 
$$	\hat{u}(w,t)=A_w\cos(wt)+B_w\sen (wt) $$
Aplicando la condición $u(x,0)=e^{-|x|}\implies \hat{u}(w,0)=\frac{2}{w^2+1}$:

$$\hat{u}(w,0)=A_w=\frac{2}{w^2+1}.$$
Entonces, tenemos: 
$$	\hat{u}(w,t)=\left(\frac{2}{w^2+1}\right)\cos(wt)+B_w\sen (wt) $$
Implica que la solución: 
$$u(w,0)=\frac{1}{2\pi}\int_{-\infty}^{\infty}\left[\left(\frac{2}{w^2+1}\right)\cos(wt)+B_w\sen (wt)\right]e^{iwx}dw.$$
\end{solution}

%---------------------------
%\bibliographystyle{apalike}
%\bibliography{sample.bib}

\end{document}