\section{Problema 3.}  Resuelva: 
$$\frac{\partial u}{\partial t}=\frac{\partial ^2 u}{\partial x^2}, \qquad 0<x<L, t>0$$
Sujeta a: $u(0,t)=0, u(L,t)=0, u(x,0)=1$.
\begin{solution}
\begin{align*}
	\Laplace{\frac{\partial u}{\partial t}}&=\Laplace{\frac{\partial ^2 u}{\partial x^2}}\\
	s\widehat{u}(x,s)-u(x,0) &= \widehat{u}''(x,s)\\
		s\widehat{u}(x,s)-1&= \widehat{u}''(x,s)\\
		 \widehat{u}''(x,s)- s\widehat{u}(x,s)&= -1
\end{align*}

\linea 

$$\widehat{u}(x,s)=\widehat{u}_H(x,s)+\widehat{u}_P(x,s)$$

\linea 

Caso homógeneo:

$$\widehat{u}''_H(x,s)-s\widehat{u}(x,s)=0$$

\linea 

Caso particular 

$$\widehat{u}''_P(x,s)-s\widehat{u}_P(x,s)=-1$$

Se propocede con coeficientes indeterminados, se propone $\widehat{u}_P=c$

\begin{align*}
	\frac{d}{dx^2}c-sc=-1 \implies 0-sc=-1\implies c=\frac{1}{s}\implies \widehat{u}_P(x,s)=\frac{1}{s}.
\end{align*}
Solución general: 
$$\widehat{u}(x,s)=Ae^{\sqrt{s}x}+Be^{-\sqrt{s}x}+\frac{1}{s}.$$

Aplicando $\widehat{u}(0,s)=0$, entonces: 

$$\widehat{u}(0,s)=A+B+\frac{1}{s}=0\implies B= -\left(\frac{As+1}{s}\right)$$
Implica que 
$$\widehat{u}(x,s)=Ae^{\sqrt{s}x}-\left(\frac{As+1}{s}\right)e^{-\sqrt{s}x}+\frac{1}{s}=2A\senh\left(\sqrt{s}x\right)+\frac{1}{s}\left(1-e^{-\sqrt{s}x}\right).$$

Aplicando $\widehat{u}(L,s)=0$, entonces: 
$$\widehat{u}(L,s)=2A\senh\left(\sqrt{s}L\right)+\frac{1}{s}\left(1-e^{-\sqrt{s}L}\right)\implies A=\frac{e^{-\sqrt{s}L}-1}{2s\senh\left(\sqrt{s}L\right)}.$$

Implica: 
$$\widehat{u}(x,s)=\left(\frac{e^{-\sqrt{s}L}-1}{s\senh\left(\sqrt{s}L\right)}\right)\senh\left(\sqrt{s}x\right)+\frac{1}{s}\left(1-e^{-\sqrt{s}x}\right).$$

Entonces, la solución es: 

$$u(x,t)=\iLaplace{\hat{u}(x,s)}=\iLaplace{\left(\frac{e^{-\sqrt{s}L}-1}{s\senh\left(\sqrt{s}L\right)}\right)\senh\left(\sqrt{s}x\right)+\frac{1}{s}\left(1-e^{-\sqrt{s}x}\right)}$$


\end{solution}