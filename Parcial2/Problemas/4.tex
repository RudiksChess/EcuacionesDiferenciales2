\section{Problema 4}

 Resuelva el problema con valores en la frontera:
$$
\frac{\partial^{2} u}{\partial t^{2}}+a^{2} \frac{\partial^{4} u}{\partial x^{4}}=0
$$
Sujeta a las condiciones:
$$
\begin{array}{c}
u(0, t)=u(l, t)=0, t>0 \\
\frac{\partial^{2} u}{\partial x^{2}}(0, t)=\frac{\partial^{2} u}{\partial x^{2}}(l, t)=0, t>0 \\
u(x, 0)=f(x), \quad 0 \leq x \leq l \\
\frac{\partial u}{\partial t}(x, 0)=g(x), \quad 0 \leq x \leq l
\end{array}
$$

\begin{solution}
Considerando: 
$$u(x,t)= X(x)\cdot T(t)$$
Entonces, substituyendo en la ecuación original:
$$XT''+a^2X^{(4)}T=0\implies a^2X^{(4)}T=-XT''\implies \frac{X^{(4)}}{-X}=\frac{T''}{a^2T}=-\lambda^2$$

\linea

La primera EDO, se define como: 
$$X^{(4)}-\lambda^2 X=0.$$
Con las condiciones,
$$u(0,t)=u(l,t)=0$$
$$\frac{\partial^{2} u}{\partial x^{2}}(0, t)=\frac{\partial^{2} u}{\partial x^{2}}(l, t)=0$$

\linita 

\fbox{Caso $\lambda^2>0$}\\ 
Proponemos una substitución, 

$$m^4-\lambda^2=0\implies (m^2-\lambda)(m^2+\lambda)=0$$
$$m=\pm \sqrt{\lambda} \qquad m=\pm i\lambda$$

Por lo que la solución general de la EDO es,
$$X(x)=Ae^{\sqrt{\lambda}x}+Be^{-\sqrt{\lambda}x}+C\cos(\sqrt{\lambda}x)+D\sin(\sqrt{\lambda}x)$$

\lineata

Con las condiciones, 
$$u(0,t)=u(l,t)=0.$$
Entonces,

\setcounter{equation}{0}
\begin{gather}
    X(0)=A+B+C=0\\
    X(l)=Ae^{\sqrt{\lambda}l}+Be^{-\sqrt{\lambda}l}+C\cos(\sqrt{\lambda}l)+D\sin(\sqrt{\lambda}l)=0
\end{gather}


\lineata


Su primera derivada, 
$$X'(x)=\sqrt{\lambda}Ae^{\sqrt{\lambda}x}-\sqrt{\lambda}Be^{-\sqrt{\lambda}x}-\sqrt{\lambda}C\sin(\sqrt{\lambda}x)+\sqrt{\lambda}D\cos(\sqrt{\lambda}x)$$

Su segunda derivada, 
$$X''(x)=\lambda Ae^{\sqrt{\lambda}x}+\lambda Be^{-\sqrt{\lambda}x}-\lambda C\cos(\sqrt{\lambda}x)-\lambda D\sin(\sqrt{\lambda}x)$$

\lineata 

Con las condiciones, 
$$\frac{\partial^{2} u}{\partial x^{2}}(0, t)=\frac{\partial^{2} u}{\partial x^{2}}(l, t)=0$$
Entonces,
\begin{gather}
    X''(0)=\lambda A+\lambda B-\lambda C=\lambda(A+B-C)=0\implies A+B-C=0
\end{gather}
$$X''(l)= \lambda Ae^{\sqrt{\lambda}l}+\lambda Be^{-\sqrt{\lambda}l}-\lambda C\cos(\sqrt{\lambda}l)-\lambda D\sin(\sqrt{\lambda}l)=0$$
$$\implies \lambda (Ae^{\sqrt{\lambda}l} +Be^{-\sqrt{\lambda}l}- C\cos(\sqrt{\lambda}l)- D\sin(\sqrt{\lambda}l))=0$$

\begin{gather}
    \implies X''(l)=Ae^{\sqrt{\lambda}l} +Be^{-\sqrt{\lambda}l}- C\cos(\sqrt{\lambda}l)- D\sin(\sqrt{\lambda}l)=0
\end{gather}

\lineata 

Entonces, por (1) y (3) tenemos que, 
$$\begin{cases}A+B+C=0\\ A+B-C=0\end{cases}\implies -C-C=0\implies -2C=0\implies C=0.$$
Por lo que, sabemos que, 
$$A+B=0 \implies B=-A.$$

Por otra parte, por (2), y (4) sabemos:
$$\begin{cases}Ae^{\sqrt{\lambda}l}+Be^{-\sqrt{\lambda}l}+C\cos(\sqrt{\lambda}l)+D\sin(\sqrt{\lambda}l)=0\\ 
Ae^{\sqrt{\lambda}l} +Be^{-\sqrt{\lambda}l}- C\cos(\sqrt{\lambda}l)- D\sin(\sqrt{\lambda}l)=0 \end{cases} $$
Ahora bien, considerando que $C=0$ y $B-A$, entonces: 

$$\begin{cases}Ae^{\sqrt{\lambda}l}+-Ae^{-\sqrt{\lambda}l}+D\sin(\sqrt{\lambda}l)=0\\ 
Ae^{\sqrt{\lambda}l} -Ae^{-\sqrt{\lambda}l}- D\sin(\sqrt{\lambda}l)=0 \end{cases}\text{ Si aplicamos una resta, entonces:} $$
$$2D\sin(\sqrt{\lambda}l)=0\implies \sin(\sqrt{\lambda}l)=0, \quad \text{en donde: $\sqrt{\lambda} l=\pi n\implies \sqrt{\lambda}=\frac{\pi n}{l}$}$$

\lineata 

Por lo tanto, 
$$X(x)=Ae^{\sqrt{\lambda}x}+Be^{-\sqrt{\lambda}x}+C\cos(\sqrt{\lambda}x)+D\sin(\sqrt{\lambda}x)$$
Como se mostró previamente, $C=0$ y $B=-A$, entonces:
$$X(x)=Ae^{\sqrt{\lambda}x}-Ae^{-\sqrt{\lambda}x}+D\sin(\sqrt{\lambda}x)= A\left(e^{\sqrt{\lambda}x}-e^{-\sqrt{\lambda}x}\right)+D\sin(\sqrt{\lambda}x)$$
$$\implies X(x)=2A\sinh(\sqrt{\lambda x})+D\sin(\sqrt{\lambda}x)$$
Considerando que $\sqrt{\lambda}= \pi n/l$, entonces: 

$$X_n(x)=A_n \sinh\left(\frac{\pi n}{l} x\right)+B_n\sin\left(\frac{\pi n}{l}x\right), \quad n=1,2,3,...$$

\linita 

\fbox{Caso $\lambda^2=0$}\\
Es decir que tenemos, 
$$X^{(4)}=0.$$
Con la solución general, 
$$X(x)=Ax^3+Bx^2+Cx+D$$

\lineata

Con las condiciones, 
$$u(0,t)=u(l,t)=0.$$
Entonces,

$$X(0)=D=0$$
$$X(l)=Al^3+Bl^2+Cl+D=Al^3+Bl^2+Cl=0$$

\lineata 

Su primera derivada, 
$$X'(x)=3Ax^2+2Bx+C$$
Su segunda derivada,
$$X''(x)=6Ax+2B$$

\lineata

Con las condiciones, 
$$\frac{\partial^{2} u}{\partial x^{2}}(0, t)=\frac{\partial^{2} u}{\partial x^{2}}(l, t)=0$$
Entonces,
$$X''(0)=B=0$$
$$X''(l)=6Al+2B=6Al=0\implies A=0.$$

Por lo tanto, $$X_0(x)=0.$$ 


\linea 

La segunda EDO, se define como: 
$$T''+(a\lambda)^2T=0.$$

Su solución general es, 
$$T(t)=A\cos(a\lambda t)+B\sin(a\lambda t)$$

En donde, $\sqrt{\lambda}=\pi n / l\implies \lambda=\left(\pi n / l\right)^2$, entonces:

$$T_n(t)=C_n\cos\left(a\frac{\pi n}{l} t\right)+D_n\sin\left(a\frac{\pi n}{l} t\right), \quad n=1,2,...$$

\linea 

Ahora bien, entonces, 

$$u(x,t)= X_nT_n=\left[A_n \sinh\left(\frac{\pi n}{l} x\right)+B_n\sin\left(\frac{\pi n}{l}x\right)\right]\cdot\left[C_n\cos\left(a\frac{\pi n}{l} t\right)+D_n\sin\left(a\frac{\pi n}{l} t\right)\right]$$
Implica: 
\begin{align*}
    u(x,t)&=\sum_{n=1}^\infty X_n\cdot T_n\\
          &= \sum_{n=1}^\infty\left[A_n \sinh\left(\frac{\pi n}{l} x\right)+B_n\sin\left(\frac{\pi n}{l}x\right)\right]\cdot\left[C_n\cos\left(a\frac{\pi n}{l} t\right)+D_n\sin\left(a\frac{\pi n}{l} t\right)\right]
\end{align*}



\end{solution}