\section{Problema 1} 
Funciones ortogonales
\begin{enumerate}
    \item  Compruebe que $f_{1}(x)=x$ y $f_{2}(x)=x^{2}$ son ortogonales en [-2, 2].
    \begin{solution}
    \begin{align}
    \begin{split}
        \langle f_1(x),f_2(x) \rangle &= \int_{-2}^2 x\cdot x^2 \diff x = \int_{-2}^2 x^3  \diff x=  \frac{1}{4} x^4 \bigg\vert_{-2}^2\\
        &= \frac{1}{4}\left[(2)^4-(-2)^4\right]= 0 
    \end{split}
    \end{align}
    $\therefore \ f_1(x),\ f_2(x)$ son ortogonales en el intervalo [-2,2]. 
    \end{solution}
    \item Determine las constantes $c_{1}$ y $c_{2}$ tales que $f_{3}(x)=x+c_{1} x^{2}+c_{2} x^{3}$ sea ortogonal a $f_{1}$ y $f_{2}$ en el mismo intervalo.
    \begin{align}
        \begin{split}
        \langle f_3(x),f_1(x) \rangle &= \int_{-2}^2 (x+c_1x^2+c_2x^3)\cdot (x)\diff x = \int_{-2}^2 (x^2+c_1x^3+c_2x^4)\diff x\\
        &= \frac{1}{3}x^3+\frac{c_1}{4}x^4+\frac{c_2}{5}x^5\bigg\vert_{-2}^2=\frac{1}{3}[(2)^3-(-2)^3]+\frac{c_2}{5}[(2)^5-(-2)^5]\\
        &= \frac{1}{3}[2^4]+\frac{c_2}{5}[2^6]
        \end{split}
        \intertext{Se sabe que $\langle f_3(x),f_1(x) \rangle=0$, entonces:}
        &\implies \frac{1}{3}[2^4]+\frac{c_2}{5}[2^6]=0 \implies c_2=-\frac{2^4\cdot 5}{3\cdot 2^6}=-\frac{5}{3\cdot2^2}= -\frac{5}{12}\\
        \begin{split}
        \langle f_3(x),f_1(x) \rangle &= \int_{-2}^2 (x+c_1x^2-\frac{5}{12}x^3)\cdot (x^2)\diff x= \int_{-2}^2 (x^3+c_1x^4-\frac{5}{12}x^5)\diff x\\
        &= \frac{1}{4}x^4+c_1\frac{1}{5}x^5-\frac{5}{60}x^6\bigg\vert_{-2}^2= \frac{c_1}{5}[(2)^5-(-2)^5]
        \end{split}
        \intertext{Nuevamente, se conoce que $\langle f_3(x),f_2(x) \rangle=0$, entonces:}
        &\implies \frac{c_1}{5}[(2)^5-(-2)^5]=0\implies c_1=0
    \end{align}
    \begin{center}
    \fbox{$c_1=0$ y $c_2=-\frac{5}{12}$}
     \end{center}
\end{enumerate}