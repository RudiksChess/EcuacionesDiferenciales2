\section{Problema 3}
Si es posible, resuelva la ecuación de Laplace $\nabla^{2} u=\frac{\partial^{2} u}{\partial x^{2}}+\frac{\partial^{2} u}{\partial y^{2}}+\frac{\partial^{2} u}{\partial z^{2}}=0,0<x<a$,
$0<y<b, 0<z<c$, sujeta a las condiciones $\frac{\partial u}{\partial x}(0, y, t)=\frac{\partial u}{\partial x}(a, y, t)=0$
$u(x, 0, z)=u(x, b, z)=u(x, y, c)=0, u(x, y, 0)=f(x, y)$

\begin{solution}
Considérese la representación gráfica del problema:
\begin{figure}[ht]
    \centering


% Pattern Info
 
\tikzset{
pattern size/.store in=\mcSize, 
pattern size = 5pt,
pattern thickness/.store in=\mcThickness, 
pattern thickness = 0.3pt,
pattern radius/.store in=\mcRadius, 
pattern radius = 1pt}
\makeatletter
\pgfutil@ifundefined{pgf@pattern@name@_po1zvyjtc}{
\pgfdeclarepatternformonly[\mcThickness,\mcSize]{_po1zvyjtc}
{\pgfqpoint{0pt}{0pt}}
{\pgfpoint{\mcSize+\mcThickness}{\mcSize+\mcThickness}}
{\pgfpoint{\mcSize}{\mcSize}}
{
\pgfsetcolor{\tikz@pattern@color}
\pgfsetlinewidth{\mcThickness}
\pgfpathmoveto{\pgfqpoint{0pt}{0pt}}
\pgfpathlineto{\pgfpoint{\mcSize+\mcThickness}{\mcSize+\mcThickness}}
\pgfusepath{stroke}
}}
\makeatother

% Pattern Info
 
\tikzset{
pattern size/.store in=\mcSize, 
pattern size = 5pt,
pattern thickness/.store in=\mcThickness, 
pattern thickness = 0.3pt,
pattern radius/.store in=\mcRadius, 
pattern radius = 1pt}
\makeatletter
\pgfutil@ifundefined{pgf@pattern@name@_pf0b9gv9h}{
\pgfdeclarepatternformonly[\mcThickness,\mcSize]{_pf0b9gv9h}
{\pgfqpoint{0pt}{0pt}}
{\pgfpoint{\mcSize+\mcThickness}{\mcSize+\mcThickness}}
{\pgfpoint{\mcSize}{\mcSize}}
{
\pgfsetcolor{\tikz@pattern@color}
\pgfsetlinewidth{\mcThickness}
\pgfpathmoveto{\pgfqpoint{0pt}{0pt}}
\pgfpathlineto{\pgfpoint{\mcSize+\mcThickness}{\mcSize+\mcThickness}}
\pgfusepath{stroke}
}}
\makeatother
\tikzset{every picture/.style={line width=0.75pt}} %set default line width to 0.75pt        

\begin{tikzpicture}[x=0.75pt,y=0.75pt,yscale=-1,xscale=1]
%uncomment if require: \path (0,300); %set diagram left start at 0, and has height of 300

%Shape: Rectangle [id:dp9023268491380846] 
\draw  [color={rgb, 255:red, 0; green, 0; blue, 0 }  ,draw opacity=1 ][pattern=_po1zvyjtc,pattern size=6pt,pattern thickness=3pt,pattern radius=0pt, pattern color={rgb, 255:red, 155; green, 155; blue, 155}][dash pattern={on 4.5pt off 4.5pt}] (315.42,55.67) -- (175.89,73.03) -- (174.73,165.81) -- (314.25,148.45) -- cycle ;
%Straight Lines [id:da12399191411618438] 
\draw    (147.43,140.71) -- (311.94,289.37) ;
\draw [shift={(313.43,290.71)}, rotate = 222.1] [color={rgb, 255:red, 0; green, 0; blue, 0 }  ][line width=0.75]    (10.93,-3.29) .. controls (6.95,-1.4) and (3.31,-0.3) .. (0,0) .. controls (3.31,0.3) and (6.95,1.4) .. (10.93,3.29)   ;
%Straight Lines [id:da17028672394227595] 
\draw    (174.43,199.71) -- (176.41,13.71) ;
\draw [shift={(176.43,11.71)}, rotate = 450.61] [color={rgb, 255:red, 0; green, 0; blue, 0 }  ][line width=0.75]    (10.93,-3.29) .. controls (6.95,-1.4) and (3.31,-0.3) .. (0,0) .. controls (3.31,0.3) and (6.95,1.4) .. (10.93,3.29)   ;
%Straight Lines [id:da1912049703634947] 
\draw    (149.39,168.25) -- (427.44,135.95) ;
\draw [shift={(429.43,135.71)}, rotate = 533.37] [color={rgb, 255:red, 0; green, 0; blue, 0 }  ][line width=0.75]    (10.93,-3.29) .. controls (6.95,-1.4) and (3.31,-0.3) .. (0,0) .. controls (3.31,0.3) and (6.95,1.4) .. (10.93,3.29)   ;
%Shape: Cube [id:dp32496726723747504] 
\draw  [fill={rgb, 255:red, 155; green, 155; blue, 155 }  ,fill opacity=0.31 ][dash pattern={on 4.5pt off 4.5pt}] (402.52,133.22) -- (315.04,55.12) -- (174.55,71.88) -- (174.77,165.21) -- (262.25,243.3) -- (402.73,226.55) -- cycle ; \draw  [dash pattern={on 4.5pt off 4.5pt}] (174.55,71.88) -- (262.04,149.97) -- (402.52,133.22) ; \draw  [dash pattern={on 4.5pt off 4.5pt}] (262.04,149.97) -- (262.25,243.3) ;
%Shape: Cube [id:dp07047973985602174] 
\draw  [fill={rgb, 255:red, 155; green, 155; blue, 155 }  ,fill opacity=0.31 ][dash pattern={on 4.5pt off 4.5pt}] (175.04,165.17) -- (262.52,243.27) -- (402.73,226.55) -- (402.52,133.22) -- (315.04,55.12) -- (174.83,71.84) -- cycle ; \draw  [dash pattern={on 4.5pt off 4.5pt}] (402.73,226.55) -- (315.25,148.45) -- (175.04,165.17) ; \draw  [dash pattern={on 4.5pt off 4.5pt}] (315.25,148.45) -- (315.04,55.12) ;

%Shape: Rectangle [id:dp3829995115583896] 
\draw  [color={rgb, 255:red, 0; green, 0; blue, 0 }  ,draw opacity=1 ][pattern=_pf0b9gv9h,pattern size=6pt,pattern thickness=3pt,pattern radius=0pt, pattern color={rgb, 255:red, 155; green, 155; blue, 155}][dash pattern={on 4.5pt off 4.5pt}] (403.21,133.13) -- (263.69,150.49) -- (262.52,243.27) -- (402.04,225.91) -- cycle ;

% Text Node
\draw (285,277.21) node [anchor=north west][inner sep=0.75pt]    {$x$};
% Text Node
\draw (156,11.21) node [anchor=north west][inner sep=0.75pt]    {$z$};
% Text Node
\draw (97.79,111.46) node [anchor=north west][inner sep=0.75pt]  [color={rgb, 255:red, 255; green, 255; blue, 255 }  ,opacity=1 ] [align=left] {11};
% Text Node
\draw (412.55,111.66) node [anchor=north west][inner sep=0.75pt]    {$y$};
% Text Node
\draw (247,238.79) node [anchor=north west][inner sep=0.75pt]   [align=left] {a};
% Text Node
\draw (405.73,229.33) node [anchor=north west][inner sep=0.75pt]   [align=left] {b};
% Text Node
\draw (160,62.79) node [anchor=north west][inner sep=0.75pt]   [align=left] {c};


\end{tikzpicture}
   \caption{Representación del problema}
\end{figure}

Se propone el método de separación de variables: 
$$u(x,y,z)=X(x)\cdot Y(y)\cdot Z(z) = X\cdot Y\cdot Z$$
Entonces: 
$$\frac{\partial^{2} u}{\partial x^{2}}+\frac{\partial^{2} u}{\partial y^{2}}+\frac{\partial^{2} u}{\partial z^{2}}=0\implies  X''YZ+ Y''XZ+Z''XY=0$$
$$\implies X''YZ+ X(Y''Z+Z''Y)=0 \implies X(Y''Z+Z''Y)= -X''YZ$$
\begin{align*}
    \implies \frac{Y''Z+Z''Y}{YZ}=-\frac{X''}{X} \implies\fbox{$\frac{Y''}{Y}+\frac{Z''}{Z}=-\frac{X''}{X}=-\lambda$}
\end{align*}

\linea 

Se considera la primera ecuación: 
$$-\frac{X''}{X}=-\lambda\implies \fbox{$X''-\lambda X=0$}$$
Con las condiciones de frontera: 
$$\frac{\partial u}{\partial x}\Big|_{x=0}=0,\quad \frac{\partial u}{\partial x}\Big|_{x=a}=0$$

Cuando $\lambda = 0$ y $\lambda<0$ las soluciones son triviales. Para el caso de $\lambda>0$, la solución es la siguiente (considerando un intervalo \textbf{finito}): 
$$X(x)= A\cosh \sqrt{\lambda}x+B\sinh\sqrt{\lambda}x$$
$$X'(x) = A\sqrt{\lambda}\sinh\sqrt{\lambda}x + B\sqrt{\lambda}\cosh\sqrt{\lambda}x  $$

Aplicando las condiciones de frontera: 

$$X'(0)= B = 0$$
$$X'(a)= A\sqrt{\lambda}\sinh \sqrt{\lambda}a$$

La relación de $\sqrt{\lambda}a$ es la siguiente: 
$$\sqrt{\lambda}a = \pi n\implies \sqrt{\lambda}=\frac{\pi n}{a}$$

Por lo tanto, la solución de la ecuación: 
$$X_n(x)= A\cosh \frac{\pi n}{a}x,\quad n=1,2,3,...$$

\linea 

Ahora bien, se tiene: $$\frac{Y''}{Y}+\frac{Z''}{Z}= -\lambda\implies \fbox{$\frac{Y''}{Y}= -\frac{Z''}{Z}-\lambda =-\mu$} $$

\linea 

Se considera la segunda ecuación diferencial 
$$\frac{Y''}{Y}=-\mu \implies \fbox{$Y'' +\mu Y=0$}$$

Con las condiciones de frontera: 
$$Y(0)=0, Y(b)=0$$

Nuevamente, cuando $\mu = 0$ y $\mu<0$ las soluciones son triviales. Para el caso de $\mu>0$, la solución es la siguiente: 
$$Y(y)= C\cos \sqrt{\mu}y+D\sin \sqrt{\mu}y$$

Aplicando las condiciones de frontera: 
$$Y(0)= C= 0 $$
$$Y(b)= D\sin \sqrt{\mu}b=0$$

La relación de $\sqrt{\mu}b$ es la siguiente: 
$$\sqrt{\mu}b=\pi m \implies \sqrt{\mu}=\frac{\pi m}{b}$$

Es decir que la solución de la ecuación es: 

$$Y_m(y) = \sin \frac{\pi m}{b}, \quad m=1,2,3,...$$

\linea 

La tercera ecuación es: 
$$-\frac{Z''}{Z}-\lambda =-\mu\implies -Z''-\lambda Z+\mu Z = 0 \implies \fbox{$Z''+(\lambda-\mu) Z = 0 $}$$
Las condiciones de frontera: 
$$ Z(c)=0$$
Nuevamente, cuando $\lambda-\mu = 0$ y $\lambda-\mu<0$ las soluciones son triviales. Para el caso de $\lambda-\mu>0$, la solución es la siguiente: 
$$Z(z)= E\cos(\lambda-\mu)z+F\sin (\lambda -\mu)z$$

Aplicando las condiciones de frontera: 
$$Z(c)= E\cos(\lambda-\mu)c+F\sin(\lambda-\mu)c=0$$
$$\implies E\cos(\lambda-\mu)c =-F\sin(\lambda-\mu)c\implies E=-\frac{F\sin(\lambda-\mu)c}{\cos(\lambda-\mu)c}$$
$$\implies E= - \tan(\lambda -\mu)c$$

Por lo que la solución es: 
$$Z(z)= -\tan(\lambda-\mu)z\cdot\cos(\lambda-\mu)z+\sin (\lambda -\mu)z$$


Se nombrará $P=\lambda -\mu$, entonces: 
$$P= \left(\frac{\pi n}{a}\right)^2-\left(\frac{\pi m}{b}\right)^2=\pi^2\left(\frac{n^2}{a^2}-\frac{m^2}{b^2}\right)$$

$$Z_{mn}(z)= -\tan \left(\frac{n^2}{a^2}-\frac{m^2}{b^2}\right)\pi^2z \cdot\cos \left(\frac{n^2}{a^2}-\frac{m^2}{b^2}\right)\pi^2z+\sin \left(\frac{n^2}{a^2}-\frac{m^2}{b^2}\right)\pi^2 z$$

Ahora, por el principio de superposición tenemos que:

\begin{align*}
    u(x,y,z) &= \sum_{m=1}^\infty \sum_{n=1}^\infty A_{mn}u_{mn}(x,y,z)\\ 
    &= A\cosh \frac{\pi n}{a}x\cdot \sin \frac{\pi m}{b}\cdot [-\tan \left(\frac{n^2}{a^2}-\frac{m^2}{b^2}\right)\pi^2z \cdot\cos \left(\frac{n^2}{a^2}-\frac{m^2}{b^2}\right)\pi^2z+\sin \left(\frac{n^2}{a^2}-\frac{m^2}{b^2}\right)\pi^2 z]
\end{align*}

Aplicando la condición inicial $z=0$: 

\begin{align*}
    f(x,y) &= \sum_{m=1}^\infty \sum_{n=1}^\infty A_{mn}u_{mn}(x,y,0)\\ 
    &= A\cosh \frac{\pi n}{a}x\cdot \sin \frac{\pi m}{b}
\end{align*}

\end{solution}