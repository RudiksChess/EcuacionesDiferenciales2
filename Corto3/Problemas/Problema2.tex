\section{Problema 2}
Resuelva la ecuación diferencial $\frac{\partial u}{\partial t}=k_{1} \frac{\partial^{2} u}{\partial x^{2}}+k_{2} \frac{\partial^{2} u}{\partial y^{2}}$, sobre el rectángulo $0<x<a$,
$0<y<b$, sujeta a las condiciones: $u(x, y, 0)=f(x, y), u(0, y, t)=0, u(a, y, t)=0$
$\frac{\partial u}{\partial y}(x, 0, t)=0, \frac{\partial u}{\partial y}(x, b, t)=0$

\begin{solution}
Se comienza planteando la representación gráfica del problema: 
\input{figurasTiks/figura_1_problema2}

Se propone utilizar el método de separación de variables: 
$$u(x,y,t)=X(x)\cdot Y(y)\cdot T(t) =X\cdot Y \cdot T $$
Es decir, sustituyendo la ecuación diferencial se tiene: 
$$XYT'= k_1 X''YT+k_2Y''XT\implies XYT'= T(k_1 X''Y+k_2Y''X)$$
$$\implies \frac{T'}{T}= \frac{k_1X''Y+k_2Y''X}{XY}\implies\fbox{$ \frac{T'}{T}=k_1\frac{X''}{X}+k_2\frac{Y''}{Y}=-\lambda$}$$

\linea 

Ahora bien, tenemos 3 casos. Dos de ellos se generan por medio de: 
$$k_1\frac{X''}{X}+k_2\frac{Y''}{Y}=-\lambda\implies k_1\frac{X''}{X} = -k_2\frac{Y''}{Y} -\lambda = -\mu $$

\linea 

Para la primera ecuación se tiene: 
$$k_1\frac{X''}{X}=-\mu\implies k_1\frac{X''}{X}+\mu X=0 \implies \fbox{$X''+\frac{\mu}{k_1}X=0$}$$
Las condiciones de frontera son:
\begin{gather*}
    X(0)= 0, \quad X(a)=0 
\end{gather*}
Nuevamente, se sabe que $\mu/k_1<0$ y $\mu/k_1=0$, se generan soluciones triviales. Por lo que solo se considerará el caso $\mu/k_1>0$. Su solución es: 
$$X(x)=A\cos \sqrt{\frac{\mu}{k_1}} x+B\sin \sqrt{\frac{\mu}{k_1}} x$$
Aplicando las condiciones de frontera: 
$$X(0)= A = 0 $$
$$X(a)=B\sin \sqrt{\frac{\mu}{k_1}}a$$
Se propone encontrar una relación para $\sqrt{\mu/k_1}a$:
$$\sqrt{\frac{\mu}{k_1}}a = \pi n\implies\sqrt{\frac{\mu}{k_1}} = \frac{\pi n}{a} $$
Por lo que la solución general es: 
$$\fbox{$X_n(x)= \sin\frac{\pi n}{a}x, \quad n=1,2,3,...$}$$
\linea 

Para la segunda ecuación se tiene: 

$$-k_2\frac{Y''}{Y}-\lambda =-\mu\implies \frac{-k_2Y''-\lambda Y}{Y}=-\mu\implies -k_2Y''-\lambda Y+\mu Y= 0$$
$$\implies k_2Y'' +\lambda Y-\mu Y=0\implies \fbox{$Y''+\frac{(\lambda-\mu)}{k_2}Y=0$}$$
Las condiciones de frontera: 
$$\frac{\partial u}{\partial y}\Big|_{y=0}=0, \quad \frac{\partial u}{\partial y}\Big|_{y=b}=0 $$
Una vez más, notamos que se generan soluciones triviales si $\frac{(\lambda-\mu)}{k_2}=0$ y $\frac{(\lambda-\mu)}{k_2}<0$. Por lo que, $\frac{(\lambda-\mu)}{k_2}>0$ es la única opción que no genera soluciones triviales y su solución es: 
$$Y(y)= C\cos\sqrt{\frac{\lambda-\mu}{k_2}}y+D\sin\sqrt{\frac{\lambda-\mu}{k_2}}y$$
$$Y'(y)= -C\sqrt{\frac{\lambda-\mu}{k_2}}\sin\sqrt{\frac{\lambda-\mu}{k_2}}y + D\sqrt{\frac{\lambda-\mu}{k_2}}\cos \sqrt{\frac{\lambda-\mu}{k_2}}y$$
Aplicando las condiciones iniciales: 
$$Y'(0)= D =0$$
$$Y'(b)= -C\sqrt{\frac{\lambda-\mu}{k_2}}\sin\sqrt{\frac{\lambda-\mu}{k_2}}b=0$$

Se propone encontrar una relación para $\sqrt{\frac{\lambda-\mu}{k_2}}$:
$$\sqrt{\frac{\lambda-\mu}{k_2}}b=\pi m\implies\sqrt{\frac{\lambda-\mu}{k_2}}= \frac{\pi m}{b}$$

Por lo que la solución general es: 
$$Y_m(y)=\cos \frac{\pi m}{b} y,\quad m=1,2,3,...$$

\linea 

Para la tercera ecuación se tiene: 
$$\implies \frac{T'}{T}=-\lambda\implies \fbox{$T'+\lambda Y=0$}$$

Excluyendo los casos triviales en donde $\lambda<0$ y $\lambda=0$, la solución para $\lambda>0$ es: 
$$Y(y)= E e^{-\lambda y}$$
Para calcular $\lambda$ se tiene de la segunda ecuación: 
$$\sqrt{\frac{\lambda-\mu}{k_2}}= \frac{\pi m}{b}\implies \lambda - \mu = k_2\left(\frac{\pi m}{b}\right)^2 \implies \lambda = \mu +k_2\left(\frac{\pi m}{b}\right)^2$$

De la primera ecuación se tiene que $\mu= k_1\left(\frac{\pi n}{a}\right)^2$, por lo que: 
$$\lambda =k_1\left(\frac{\pi n}{a}\right)^2 +k_2\left(\frac{\pi m}{b}\right)^2 $$
Por lo tanto, la solución de la ecuación es: 
\begin{align*}
    Y_{mn}(y) =\exp\left(- \left(k_1\left(\frac{\pi n}{a}\right)^2 +k_2\left(\frac{\pi m}{b}\right)^2\right)t\right)\\
    \fbox{$Y_{mn}(y) = \exp\left(- \pi^2\left(k_1\left(\frac{ n}{a}\right)^2 +k_2\left(\frac{ m}{b}\right)^2\right)t\right)$}
\end{align*}

\linea 

Ahora bien, la solución general de $u$ para $m,n=1,2,3,...$:
\begin{align*}
    u(x,y,t)_{nm} &= X(x)\cdot Y(y)\cdot T(y)\\
             &= \sin\frac{\pi n}{a}x \cdot \cos \frac{\pi m}{b} y \cdot \exp\left(- \pi^2\left(k_1\left(\frac{ n}{a}\right)^2 +k_2\left(\frac{ m}{b}\right)^2\right)t\right)
\end{align*}

\linea 

Aplicando el principio de superposición: 
\begin{align*}
    u(x,y,t) &= \sum_{m=1}^\infty\sum_{n=1}^\infty A_{mn}u_{mn}(x,y,t)\\
    &= \sum_{m=1}^\infty\sum_{n=1}^\infty A_{mn}\sin\frac{\pi n}{a}x \cdot \cos \frac{\pi m}{b} y \cdot \exp\left(- \pi^2\left(k_1\left(\frac{ n}{a}\right)^2 +k_2\left(\frac{ m}{b}\right)^2\right)t\right)
\end{align*}

\linea 

Aplicando la condición inicial en donde $t=0$, se tiene: 
\begin{align*}
    f(x,y) &= \sum_{m=1}^\infty\sum_{n=1}^\infty A_{mn}u_{mn}(x,y,t)\\
    &= \sum_{m=1}^\infty\sum_{n=1}^\infty A_{mn}\sin\frac{\pi n}{a}x \cdot \cos \frac{\pi m}{b} y
\end{align*}

\linea 

Ahora bien, tomando como referencia la ortogonalidad de los senos  y cosenos, se propone multiplicar por $\sin (k \pi x / a) \cos (l \pi y / b)$ (considerando $l$ y $k$ como $n$ y diferenciarlos) e integrar en $x$ de 0 a $a$, y en $y$ de 0 a $b$, es decir: 
$$
\begin{aligned}
& \int_{x=0}^{a} \int_{y=0}^{b} f(x, y) \sin \left(\frac{k \pi x}{a}\right) \cos \left(\frac{l \pi y}{b}\right) d y d x \\
=& \sum_{m=1}^{\infty} \sum_{n=1}^{\infty} A_{m n} \int_{x=0}^{a} \int_{y=0}^{b} \sin \left(\frac{n \pi x}{a}\right) \sin \left(\frac{k \pi x}{a}\right) \cos \left(\frac{m \pi y}{b}\right) \cos \left(\frac{l \pi y}{b}\right) d y d x \\
=& \sum_{m=1}^{\infty} \sum_{n=1}^{\infty} A_{m n} \int_{x=0}^{a} \sin \left(\frac{n \pi x}{a}\right) \sin \left(\frac{k \pi x}{a}\right) d x \int_{y=0}^{b} \cos \left(\frac{m \pi y}{b}\right) \cos \left(\frac{l \pi y}{b}\right) d y \\
=& \sum_{m=1}^{\infty} \sum_{n=1}^{\infty} A_{m n} \frac{a}{2} \delta_{n k} \frac{b}{2} \delta_{m l} \\
=& A_{l k} \frac{ab}{4}
\end{aligned}
$$
Se tiene, entonces:
$$
A_{m n}=\frac{4}{ab} \int_{x=0}^{a} \int_{y=0}^{b} f(x, y) \sin \left(\frac{n \pi x}{a}\right) \cos \left(\frac{m \pi y}{b}\right) d y d x
$$

Finalmente, la solución: 
\begin{align*}
    u(x,y,t) &=
    \sum_{m=1}^\infty\sum_{n=1}^\infty \frac{4}{ab} \int_{x=0}^{a} \int_{y=0}^{b} f(x, y) \sin \left(\frac{n \pi x}{a}\right) \cos \left(\frac{m \pi y}{b}\right) d y d x\sin\frac{\pi n}{a}x \cdot \cos \frac{\pi m}{b} y \\ &\cdot \exp\left(- \pi^2\left(k_1\left(\frac{ n}{a}\right)^2 +k_2\left(\frac{ m}{b}\right)^2\right)t\right)
\end{align*}
\end{solution}